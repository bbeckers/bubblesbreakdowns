

    \filetitle{autoexogenise}{Get or set variable/shock pairs for use in autoexogenised simulation plans}{model/autoexogenise}

	\paragraph{Syntax fo getting autoexogenised variable/shock pairs}
 
 \begin{verbatim}
 A = autoexogenise(M)
 \end{verbatim}
 
 \paragraph{Syntax fo setting autoexogenised variable/shock pairs}
 
 \begin{verbatim}
 M = autoexogenise(M,A)
 \end{verbatim}
 
 \paragraph{Input arugments}
 
 \begin{itemize}
 \item
   \texttt{M} {[} model {]} - Model object.
 \item
   \texttt{A} {[} struct \textbar{} empty {]} - Database with each field
   representing a variable/shock pair, A.Variable\_Name = `Shock\_Name',
   that can be used in building \href{plan/Contents}{simulation plans} by
   the plan function \href{plan/autoexogenise}{\texttt{autoexogenise}}.
 \end{itemize}
 
 \paragraph{Output arguments}
 
 \begin{itemize}
 \item
   \texttt{M} {[} model {]} - Model object with updated definitions of
   autoexogenised variable/shock pairs.
 \end{itemize}
 
 \paragraph{Description}
 
 Whenever you set the autoexogenised variable/shock pairs, the previously
 assigned pairs are removed, and replaced with the new ones in
 \texttt{A}. In other words, the new pairs are not added to the existing
 ones, the replace them.
 
 \paragraph{Example}


