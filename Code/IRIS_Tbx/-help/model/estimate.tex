

    \filetitle{estimate}{Estimate model parameters by optimising selected objective function}{model/estimate}

	\paragraph{Syntax}
 
 \begin{verbatim}
 [PEst,Pos,Cov,Hess,M,V,Delta,PDelta] = estimate(M,D,Range,Est,...)
 [PEst,Pos,Cov,Hess,M,V,Delta,PDelta] = estimate(M,D,Range,Est,SPr,...)
 \end{verbatim}
 
 \paragraph{Input arguments}
 
 \begin{itemize}
 \item
   \texttt{M} {[} struct {]} - Model object.
 \item
   \texttt{D} {[} struct \textbar{} cell {]} - Input database or datapack
   from which the measurement variables will be taken.
 \item
   \texttt{Range} {[} struct {]} - Date range.
 \item
   \texttt{Est} {[} struct {]} - Database with the list of paremeters
   that will be estimated, and the parameter prior specifications (see
   below).
 \item
   \texttt{SPr} {[} empty \textbar{} systempriors {]} - System priors
   object, \href{systempriors/Contents}{\texttt{systempriors}}.
 \end{itemize}
 
 \paragraph{Output arguments}
 
 \begin{itemize}
 \item
   \texttt{PEst} {[} struct {]} - Database with point estimates of
   requested parameters.
 \item
   \texttt{Pos} {[} poster {]} - Posterior,
   \href{poster/Contents}{\texttt{poster}}, object; this object also
   gives you access to the value of the objective function at optimum or
   at any point in the parameter space, see the \url{poster/eval}
   function.
 \item
   \texttt{Cov} {[} numeric {]} - Approximate covariance matrix for the
   estimates of parameters with slack bounds based on the asymptotic
   Fisher information matrix (not on the Hessian returned from the
   optimisation routine).
 \item
   \texttt{Hess} {[} cell {]} - \texttt{Hess\{1\}} is the total hessian
   of the objective function; \texttt{Hess\{2\}} is the contributions of
   the priors to the hessian.
 \item
   \texttt{M} {[} model {]} - Model object solved with the estimated
   parameters (including out-of-likelihood parameters and common variance
   factor).
 \end{itemize}
 
 The remaining three output arguments, \texttt{V}, \texttt{Delta},
 \texttt{PDelta}, are the same as the \url{model/loglik} output arguments
 of the same names.
 
 \paragraph{Options}
 
 \begin{itemize}
 \item
   \texttt{'chkSstate='} {[} \texttt{true} \textbar{}
   \emph{\texttt{false}} \textbar{} cell {]} - Check steady state in each
   iteration; works only in non-linear models.
 \item
   \texttt{'evalFrfPriors='} {[} \emph{\texttt{true}} \textbar{}
   \texttt{false} {]} - In each iteration, evaluate frequency response
   function prior density, and include it to the overall objective
   function to be optimised.
 \item
   \texttt{'evalLik='} {[} \emph{\texttt{true}} \textbar{} \texttt{false}
   {]} - In each iteration, evaluate likelihood (or another data based
   criterion), and include it to the overall objective function to be
   optimised.
 \item
   \texttt{'evalPPriors='} {[} \emph{\texttt{true}} \textbar{}
   \texttt{false} {]} - In each iteration, evaluate parameter prior
   density, and include it to the overall objective function to be
   optimised.
 \item
   \texttt{'evalSPriors='} {[} \emph{\texttt{true}} \textbar{}
   \texttt{false} {]} - In each iteration, evaluate system prior density,
   and include it to the overall objective function to be optimised.
 \item
   \texttt{'filter='} {[} cell \textbar{} \emph{empty} {]} - Cell array
   of options that will be passed on to the Kalman filter including the
   type of objective function; see help on \url{model/filter} for the
   options available.
 \item
   \texttt{'initVal='} {[} \texttt{model} \textbar{}
   \emph{\texttt{struct}} \textbar{} struct {]} - If \texttt{struct} use
   the values in the input struct \texttt{Est} to start the iteration; if
   \texttt{model} use the currently assigned parameter values in the
   input model, \texttt{M}.
 \item
   \texttt{'maxIter='} {[} numeric \textbar{} \emph{\texttt{500}} {]} -
   Maximum number of iterations allowed.
 \item
   \texttt{'maxFunEvals='} {[} numeric \textbar{} \emph{\texttt{2000}}
   {]} - Maximum number of objective function calls allowed.
 \item
   \texttt{'noSolution='} {[} \emph{\texttt{'error'}} \textbar{}
   \texttt{'penalty'} {]} - Specifies what happens if solution or steady
   state fails to solve in an iteration: \texttt{'error='} stops the
   execution with an error message, \texttt{'penalty='} returns an
   extremely low value of the likelihood.
 \item
   \texttt{'optimSet='} {[} cell \textbar{} \emph{empty} {]} - Cell array
   used to create the Optimization Toolbox options structure; works only
   with the option \texttt{'optimiser='} \texttt{'default'}.
 \item
   \texttt{'refresh='} {[} \emph{\texttt{true}} \textbar{} \texttt{false}
   {]} - Refresh dynamic links in each iteration.
 \item
   \texttt{'solve='} {[} \emph{\texttt{true}} \textbar{} \texttt{false}
   {]} - Re-compute solution in each iteration.
 \item
   \texttt{'optimiser='} {[} \emph{\texttt{'default'}} \textbar{}
   \texttt{'pso'} \textbar{} cell \textbar{} function\_handle {]} -
   Minimisation procedure.
 
   \begin{itemize}
   \item
     \texttt{'default'}: The Optimization Toolbox function
     \texttt{fminunc} or \texttt{fmincon} will be called depending on the
     presence or absence of lower and/or upper bounds.
   \item
     \texttt{'pso'}: The Particle Swarm Optimizer will be called; use the
     option \texttt{'pso='} to specify further options to control the
     optimizer (see Options for Particle Swarm Optimizer below).
   \item
     function\_handle or cell: Enter a function handle to your own
     optimisation procedure, or a cell array with a function handle and
     additional input arguments (see below).
   \end{itemize}
 \item
   \texttt{'sstate='} {[} \texttt{true} \textbar{} \emph{\texttt{false}}
   \textbar{} cell \textbar{} function\_handle {]} - Re-compute steady
   state in each iteration. You can specify a cell array with options for
   the \texttt{sstate} function, or a function handle whose behaviour is
   described below.
 \item
   \texttt{'tolFun='} {[} numeric \textbar{} \emph{\texttt{1e-6}} {]} -
   Termination tolerance on the objective function.
 \item
   \texttt{'tolX='} {[} numeric \textbar{} \emph{\texttt{1e-6}} {]} -
   Termination tolerance on the estimated parameters.
 \end{itemize}
 
 \paragraph{Options for Particle Swarm Optimizer}
 
 The following options can be specified through the main option
 \texttt{'optimset='} when \texttt{'optimiser=pso'}.
 
 \begin{itemize}
 \item
   \texttt{'cognitiveAttraction='} {[} numeric \textbar{}
   \emph{\texttt{0.5}} {]} - Scalar between \texttt{0} and \texttt{1} to
   control the relative attraction to the best location a particle can
   remember.
 \item
   \texttt{'constrBoundary='} {[} \texttt{absorb} \textbar{}
   \emph{\texttt{reflect}} \textbar{} \texttt{soft} {]} - Controls the
   way imposed constraints are handled when violated.
 
   \begin{itemize}
   \item
     \texttt{'soft'}: Particles are allowed to travel outside the bounds
     but get bad fitness function (likelihood) values when they do;
   \item
     \texttt{'reflect'}: Particle velocity is changed such that when the
     particle encounters the bound its velocity is changed to effectively
     make it bounce off of the boundary;
   \item
     \texttt{'absorb'}: Particles hit the bound and stay at the bound
     until attracted elsewhere because its velocity is set to zero.
   \end{itemize}
 \item
   \texttt{'display='} {[} \texttt{'off'} \textbar{} \texttt{'final'}
   \textbar{} \emph{\texttt{'iter'}} {]} - Level of display in order of
   increasing verbosity; \texttt{'iter'} will only produce output at most
   \texttt{'updateInterval='} seconds.
 \item
   \texttt{'fitnessLimit='} {[} numeric \textbar{} \emph{\texttt{-Inf}}
   {]} - Algorithm will terminate when a function value this low is
   encountered.
 \item
   \texttt{'generations='} {[} numeric \textbar{} \emph{\texttt{1000}}
   {]} - Positive integer describing the maximum length of swarm
   evolution.
 \item
   \texttt{'hybridFcn='} {[} \texttt{true} \textbar{}
   \emph{\texttt{false}} \textbar{} \texttt{'fmincon'} \textbar{}
   \texttt{'fminunc'} \textbar{} cell {]} - Run a second stage
   optimization after PSO (only available with the Optimization Tbx
   installed):
 
   \begin{itemize}
   \item
     \texttt{false}: No second stage optimization, run the particle swarm
     only.
   \item
     \texttt{true}: After PSO, run either \texttt{fminunc} or
     \texttt{fmincon}, the Optimization Toolbox routines, depending on
     the presence or absence of lower and upper bounds on estimated
     parameters.
   \item
     \texttt{'fminunc'}, \texttt{'fmincon'}: After PSO, run the specified
     Optimization Toolbox routine.
   \item
     cell: A cell array in which the first argument specifies the
     function as previously and the second argument contains the options
     structure for that function; for instance
     \texttt{\{@fmincon,optimset('Display','iter')\}}.
   \end{itemize}
 \item
   \texttt{'includeInitialValue='} {[} \emph{\texttt{true}} \textbar{}
   \texttt{false} {]} - Include the initial vector of parameters in the
   initial population.
 \item
   \texttt{'initialPopulation=}' {[} numeric \textbar{} \emph{empty} {]}
   - An NPar-by-NPop array containing the initial distribution of
   particles, where NPar is the number of estimated parameters, and NPop
   is the size of population. If empty, a population will be created
   containing the initial parameter vector and the rest of the particles
   will be randomly generated according to \texttt{'popInitRange='}. Use
   the option \texttt{'includeInitialValue=' false} oo exclude the
   initial value from the initial population so that the entire
   population is randomly generated.
 \item
   \texttt{'socialAttraction='} {[} numeric \textbar{}
   \emph{\texttt{1.25}} {]} - Positive scalar to control the relative
   attraction of each particle to the best location they have heard about
   from other particles.
 \item
   \texttt{'plotFcns='} {[} cell \textbar{} \emph{empty} {]} - Cell array
   of function handles to functions which accept
   \texttt{(options,state,flag)} values as input arguments. The only
   built-in general-purpose plotting function is
   \texttt{@optim.scoreDiversity}.
 \item
   \texttt{'populationSize='} {[} numeric \textbar{} \emph{\texttt{40}}
   {]} - Positive integer which determines the number of particles in the
   swarm.
 \item
   \texttt{'popInitRange='} {[} numeric \textbar{} \emph{empty} {]} - A
   2-by-NPar array which sets the range over which the initial population
   will be distributed, where NPar is the number of estimated parameters,
   or a 2-by-1 array with the range for all parameters. If empty and
   \texttt{'PopInitRange='} is not set, the upper and lower bounds will
   be used if both are finite. If either of the bounds are infinite, the
   range will be \texttt{{[}0;1{]}}.
 \item
   \texttt{'stallGenLimit='} {[} numeric \textbar{} \emph{\texttt{100}}
   {]} - Maximum number of swarm iterations which result in no
   improvement in the fitness function (likelihood) value before the
   algorithm terminates.
 \item
   \texttt{'timeLimit='} {[} numeric \textbar{} \emph{\texttt{Inf}} {]} -
   Maximum running time in seconds.
 \item
   \texttt{'tolCon='} {[} numeric \textbar{} \emph{\texttt{1e-6}} {]} -
   Largest tolerated constraint violation.
 \item
   \texttt{'tolFun='} {[} numeric \textbar{} \emph{\texttt{1e-6}} {]} -
   Function tolerance; when the change in the best fitness function value
   (likelihood) improvement per generation falls below this value the
   algorithm will terminate.
 \item
   \texttt{'velocityLimit='} {[} numeric \textbar{} \emph{\texttt{Inf}}
   {]} - Positive scalar to bound particle intertia from above.
 \item
   \texttt{'updateInterval='}* {[} numeric \textbar{} \texttt{5} {]} -
   Minimum length of time in seconds which must pass before new command
   window output will be produced.
 \item
   \texttt{'useParallel='} {[} \texttt{true} \textbar{}
   \emph{\texttt{false}} {]} - Use a \texttt{parfor} loop which requires
   you already have a \texttt{matlabpool} open. Overhead is slightly
   higher for constrained problems than unconstrained problems.
 \end{itemize}
 
 \paragraph{Description}
 
 In the input parameter database, \texttt{E}, you can provide the
 following four specifications for each parameter:
 
 \begin{verbatim}
 E.parameter_name = {start,lower,upper,logprior}
 \end{verbatim}
 
 where \texttt{start} is the value from which the numerical optimisation
 will start, \texttt{lower} is the lower bound, \texttt{upper} is the
 upper bound, and \texttt{logprior} is a function handle expected to
 return the log of the prior density. You can use the
 \href{logdist/Contents}{\texttt{logdist}} package to create function
 handles for some of the basic prior distributions.
 
 You can use \texttt{NaN} for \texttt{start} if you wish to use the value
 currently assigned in the model object. You can use \texttt{-Inf} and
 \texttt{Inf} for the bounds, or leave the bounds empty or not specify
 them at all. You can leave the prior distribution empty or not specify
 it at all.
 
 \subparagraph{User-supplied optimisation (minimisation) routine}
 
 You can supply a function handle to your own minimisation routine
 through the option \texttt{'optimiser='}. This routine will be used
 instead of the Optim Tbx's \texttt{fminunc} or \texttt{fmincon}
 functions. The user-supplied function is expected to take at least five
 input arguments and return three output arguments:
 
 \begin{verbatim}
 [PEst,ObjEst,Hess] = yourminfunc(F,P0,PLow,PHigh,Opt)
 \end{verbatim}
 
 with the following input arguments:
 
 \begin{itemize}
 \item
   \texttt{F} is a function handle to the function minimised;
 \item
   \texttt{P0} is a 1-by-N vector of initial parameter values;
 \item
   \texttt{PLow} is a 1-by-N vector of lower bounds (with \texttt{-Inf}
   indicating no lower bound);
 \item
   \texttt{PHigh} is a 1-by-N vector of upper bounds (with \texttt{Inf}
   indicating no upper bounds);
 \item
   \texttt{Opt} is an Optim Tbx style struct with the optimisation
   settings (tolerance, number of iterations, etc); of course you may
   simply ignore this information and leave the input argument unused.
 \end{itemize}
 
 and the following output arguments:
 
 \begin{itemize}
 \item
   \texttt{PEst} is a 1-by-N vector of estimated parameters;
 \item
   \texttt{ObjEst} is the value of the objective function at optimum;
 \item
   \texttt{Hess} is a N-by-N approximate Hessian matrix at optimum.
 \end{itemize}
 
 If you need to use extra input arguments in your minimisation function,
 enter a cell array instead of a plain function handle:
 
 \begin{verbatim}
 {@yourminfunc,Arg1,Arg2,...}
 \end{verbatim}
 
 In that case, the optmiser will be called the following way:
 
 \begin{verbatim}
 [PEst,ObjEst,Hess] = yourminfunc(F,P0,PLow,PHigh,Opt,Arg1,Arg2,...)
 \end{verbatim}
 
 \subparagraph{User-supplied steady-state solver}
 
 You can supply a function handle to your own steady state solver (i.e.~a
 function that finds the steady state for given parameters) through the
 \texttt{'sstate='} option.
 
 The function is expected to take one input argument, the model object
 with newly assigned parameters, and return at least two output
 arguments, the model object with a new steady state (or balanced-growth
 path) and a success flag. The flag is \texttt{true} if the steady state
 has been successfully computed, and \texttt{false} if not:
 
 \begin{verbatim}
 [M,Success] = yoursstatesolver(M)
 \end{verbatim}
 
 It is your responsibility to add the growth characteristics if some of
 the model variables drift over time. In other words, you need to take
 care of the imaginary parts of the steady state values in the model
 object returned by the solver.
 
 \paragraph{Example}


