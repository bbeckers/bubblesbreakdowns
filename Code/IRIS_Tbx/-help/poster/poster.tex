

    \filetitle{poster}{Posterior objects and functions}{poster/poster}

	Posterior objects, \texttt{poster}, are used to evaluate the behaviour
 of the posterior dsitribution, and to draw model parameters from the
 posterior distibution.
 
 Posterior objects are set up within the \url{model/estimate} function
 and returned as the second output argument - the set up and
 initialisation of the posterior object is fully automated in this case.
 Alternatively, you can set up a posterior object manually, by setting
 all its properties appropriately.
 
 Poster methods:
 
 \paragraph{Constructor}
 
 \begin{itemize}
 \item
   \href{poster/poster}{\texttt{poster}} - Posterior objects and
   functions.
 \end{itemize}
 
 \paragraph{Evaluating posterior density}
 
 \begin{itemize}
 \item
   \href{poster/arwm}{\texttt{arwm}} - Adaptive random-walk Metropolis
   posterior simulator.
 \item
   \href{poster/eval}{\texttt{eval}} - Evaluate posterior density at
   specified points.
 \end{itemize}
 
 \paragraph{Chain statistics}
 
 \begin{itemize}
 \item
   \href{poster/stats}{\texttt{stats}} - Evaluate selected statistics of
   ARWM chain.
 \end{itemize}
 
 \paragraph{Getting on-line help on model functions}
 
 \begin{verbatim}
 help poster
 help poster/function_name
 \end{verbatim}


