

    \filetitle{highlight}{Highlight range in graph}{report/highlight}

	\paragraph{Syntax}
 
 \begin{verbatim}
 P.highlight(Caption,Range,...)
 \end{verbatim}
 
 \paragraph{Input arguments}
 
 \begin{itemize}
 \item
   \texttt{P} {[} struct {]} - Report object created by the
   \href{report/new}{\texttt{report.new}} function.
 \item
   \texttt{Caption} {[} char {]} - Caption used to annotate the
   highlighted area.
 \item
   \texttt{Range} {[} cell \textbar{} numeric {]} - Date range, or a cell
   array of ranges, that will be highlighted.
 \end{itemize}
 
 \paragraph{Options}
 
 \begin{itemize}
 \item
   \texttt{'hPosition='} {[} \texttt{'bottom'} \textbar{}
   \texttt{'middle'} \textbar{} \emph{\texttt{'top'}} {]} - (Inheritable
   from parent objects) Horizontal position of the caption.
 \item
   \texttt{'vPosition='} {[} \texttt{'centre'} \textbar{} \texttt{'left'}
   \textbar{} \emph{\texttt{'right'}} {]} - (Inheritable from parent
   objects) Vertical position of the caption relative to the edges of the
   highlighted area.
 \end{itemize}
 
 \paragraph{Generic options}
 
 See help on \href{report/Contents}{generic options} in report objects.
 
 \paragraph{Description}
 
 \paragraph{Example}


