

    \filetitle{compile}{Compile an m-file function based on a steady-state file}{sstate/compile}

	\paragraph{Syntax}
 
 \begin{verbatim}
 compile(S)
 compile(S,Fname,...)
 \end{verbatim}
 
 \paragraph{Input arguments}
 
 \begin{itemize}
 \item
   \texttt{S} {[} sstate {]} - Sstate object built on a steady-state
   file.
 \item
   \texttt{Fname} {[} char \textbar{} empty {]} - Filename of the
   compiled m-file function; if not specified or empty the original
   steady-state filename will be used with an '.m' extension.
 \end{itemize}
 
 \paragraph{Options}
 
 \begin{itemize}
 \item
   \texttt{'excludeZero='} {[} \texttt{true} \textbar{}
   \emph{\texttt{false}} {]} - Automatically detect and exclude zero
   solutions in blocks that result in multiple solutions.
 \item
   \texttt{'deleteSymbolicMfiles='} {[} \emph{\texttt{true}} \textbar{}
   \texttt{false} {]} - Delete auxiliary m-files created to call the
   Symbolic Math toolbox.
 \item
   \texttt{'end='} {[} numeric \textbar{} char \textbar{}
   \emph{\texttt{Inf}} {]} - Compile the steady-state m-file function
   only up to this block.
 \item
   \texttt{'simplify='} {[} numeric \textbar{} \emph{\texttt{Inf}} {]} -
   The minimum length for a symbolic expression to be simplified using
   the \texttt{simplify} function; Inf means no expressions will undergo
   simplification.
 \item
   \texttt{'start='} {[} numeric \textbar{} char \textbar{} \emph{1} {]}
   - Compile the steady-state m-file function starting from this block.
 \item
   \texttt{'symbolic='} {[} \emph{\texttt{true}} \textbar{}
   \texttt{false} {]} - Call the Symbolic Math toolbox to solve blocks
   marked with a \href{sstatelang/symbolic}{\texttt{!symbolic}} keyword;
   otherwise, all blocks will be solved numerically regardless of their
   specification.
 \end{itemize}
 
 \paragraph{Description}
 
 \paragraph{Example}


